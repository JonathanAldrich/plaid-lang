\documentclass[12pt]{article}
\usepackage{fullpage,cmu-titlepage2}
\usepackage{times}
\usepackage{excludeonly}

\usepackage[T1]{fontenc}

\usepackage{amssymb}
\usepackage{listings}
\usepackage{mathpartir}
\usepackage{graphicx}
\usepackage{tabularx}
\usepackage[release]{PlaidDefinitions}
\usepackage{datetime}
\usepackage{natbib}

\newcommand{\singlespace}{\renewcommand{\baselinestretch}{1.0}\normalsize}

\renewcommand{\cite}{\citep}


%%%%%%%%%%%%%%%%%%%%%%%%%%%%%%%%%%%%%%%%%%%%%%%%%%%%%%%%%%%%%%%%%%%%%%%%%%%%%

\title{The Plaid Language:\\
Typed Core Specification\\
\vspace{2ex}
version 0.4.0\\
\vspace{2ex}
}

\author{Jonathan Aldrich \and Nels E. Beckman \and Robert Bocchino \and Karl Naden \and Darpan Saini \and Sven Stork \and Joshua Sunshine}


\date{\monthname~\the \year}

\abstract{
Plaid is an object oriented programming language built on two paradigms.  First, Plaid
is {\it typestate-oriented}. Programmers can directly encode the {\it abstract states}
of objects and use the {\it state change operator} to change the state, interface, and representation
of an object at runtime.  Second, Plaid's type system is {\it permission-based}.  The type of each reference 
includes an {\it access permission} which dictates how the reference can be used and characterizes
the permissions to other aliases of the same object.  Plaid leverages permissions 
when tracking the abstract state of references during typechecking.  Permissions are
also used to infer code that can be safely run in parallel.
This document defines the core of the Plaid language, including its source syntax, 
the semantics of operations involving abstract states, and a type system.
}





\keywords{programming language, typestate, Plaid, gradual typing, permissions}

\trnumber{CMU-ISR-12-103}


\support{This research was supported by DARPA grant HR00110710019, CMU|Portugal Aeminium grant CMU-PT/SE/0038/2008, NSF grants CCF-0811592 and CCF-1116907, and grant \#1019343 to the Computing Research Association for the
CIFellows Project.}


\begin{document}

\renewcommand*{\thepage}{title-\arabic{page}} 
\maketitle
\renewcommand*{\thepage}{\arabic{page}} 




%%%%%%%%%%%%%%%%%%%% MAIN TEXT %%%%%%%%%%%%%%%%%%%%%%%%%%%%%%

\section{Conventions}

This document uses the same grammar definition conventions as the Java
Language Specification, Third Edition (JLS) ~\cite{gosling2005}.  Those conventions
are described in chapter 2 of the JLS and are not repeated here.




\section{Lexical Structure}

The lexical structure of Plaid is largely borrowed from Java, but there are 
significant differences.  Specifically:

\begin{itemize}

\item Plaid uses the same definitions of line terminators as Java (JLS
  section 3.4), the same input elements and tokens (JLS section 3.5)
  except for a different keyword list, and the same definition of
  whitespace (JLS section 3.6).

\item Plaid uses the same definition of comments as Java (JLS section 3.7).

\item Plaid uses the same definition of identifiers as Java (JLS section 3.8).

\item Plaid literals are based on Java literals (JLS section
3.10), but there are several substantial differences. Plaid 
string literals are the same as Java string literals. 
Plaid integer literals are of arbitrary length. Plaid 
rational literals are like Java double literals but are of 
arbitrary length. Furthermore, Boolean objects named 
\texttt{true} and \texttt{false} exist in the standard library,
but unlike in Java these are not keywords in Plaid. No other 
Java literals are currently supported, but future versions 
of Plaid will support all Java literals except the null literal.
  
\item Plaid uses the same definition of Separators as Java (JLS section 3.11).

\end{itemize}


\subsection{Keywords}

The following character sequences, formed from ASCII letters, are reserved
for use as \textit{keywords} and cannot be used as identifiers:

\begin{quote}
\ntermdef{Keyword} \oneof 
\end{quote}
\[
  \begin{array}{cccccc}
  \keyw{atomic}
  & \keyw{callonce}
  & \keyw{case} 
  & \keyw{default}
  & \keyw{dyn} 
  \\ 
  \keyw{dynamic}
  & \keyw{exclusive} 
  & \keyw{fn}
  & \keyw{freeze}
  & \keyw{full} 
  \\
  \keyw{group}  
  &\keyw{immutable}
  & \keyw{import}
  & \keyw{local}
  & \keyw{match}
  \\
  \keyw{method}
  & \keyw{mutable}
  &\keyw{new}
  & \keyw{none}
  & \keyw{of}
  \\
  \keyw{override} 
  & \keyw{package}
  & \keyw{pure}
  & \keyw{readonly}
  & \keyw{remove}
  \\
  \keyw{rename}
  & \keyw{requires}
  &\keyw{shared}
  & \keyw{split}  
  & \keyw{state}
  \\
  \keyw{stateval}
  & \keyw{this}
  & \keyw{top}
  & \keyw{type} 
  & \keyw{unique} 
  \\ 
  \keyw{unpack}
  & \keyw{val}
  & \keyw{var}
  & \keyw{void}
  & \keyw{with}
  \end{array}
\]

%\TODO{The list above is only a current estimate, this will change as
%  we get the whole language defined.  Should also put the above into
%a nice table as in the JLS.}

%Note that \keyw{true} and \keyw{false} are actually boolean literals,
%as in Java.





\subsection{Operators}

We first define operator characters as follows:

\begin{quote}

\ntermdef{OperatorChar} \oneof

\defspace \texttt{= < > ! $\sim$ ? : \& | + - * / \^{} \%}

\end{quote}

Now an operator is a sequence of operator characters:

\begin{quote}

\ntermdef{Operator}

\defspace \nterm{OperatorChar}

\defspace \nterm{OperatorChar} \nterm{Operator}

\end{quote}

The exception to the grammar above is that the character sequences
\texttt{=}, \texttt{=>}, \texttt{<{}<-}, \texttt{>{}>} and \texttt{<-} have
other meanings in the language and may not be used as operators.
Furthermore, operators containing the comment sequences \texttt{/*}
or \texttt{//} may not be used as operators.


\section{Statements and Expressions}

\subsection{Exceptions}

Several locations in this document refer to an exception being thrown.
The semantics of an exception being thrown is that the application
halts with a run-time error.  Future versions of this document will
define facilities for propagating and catching exceptions.

\subsection{Statements}


\begin{quote}
\ntermdef{Stmt}

\defspace \nterm{Expr}

\defspace \nterm{VarDecl}

\defspace \nterm{StateValDecl}


\ntermdef{VarDecl}

\defspace \nterm{Specifier} \opt{\nterm{Type}} \nterm{Identifier} = \nterm{Expr}

\ntermdef{StateValDecl}

\defspace \keyw{stateval} \nterm{Identifier} \nterm{StateBinding}


\ntermdef{Specifier}

\defspace \keyw{val}

\defspace \keyw{var}

\end{quote}

Statements are either expressions, or variable declarations.  A
variable declaration must include an initial value.  Object variables are
declared with the \keyw{val} or \keyw{var} keyword; the former
indicates a final let binding, whereas the latter indicates a
assignable variable that can be updated.
State variables are declared with the \keyw{stateval} keyword. 

An optional type may be given for variable declarations.  If the type
is omitted for a \keyw{val} declaration, then it is inferred to have the
structure of the initializing expression and the permission that
is the default for that structure. If no type is given for a \keyw{var} 
declaration the variable is considered to have type \keyw{dynamic}.

Statements evaluate to values, based on the expression in the
statement or the value of the initializer for the variable.  The last
statement in a sequence is used for the return value of a method or
the result of a block.

\subsection{Expressions}

\begin{quote}

\ntermdef{Expr}


\defspace \keyw{fn}   \opt{\nterm{MetaArgsSpec}}(\opt{\nterm{Args}})
              \opt{[\nterm{Args}]}
              => \nterm{Expr}

\defspace \nterm{Expr1}

\end{quote}

A first-class function includes standard polymorphic and parameter
argument declarations. The optional arguments surrounded by
  [] support specifying types for variables that are
  currently in scope, including any changes to the types
  of these variables made by a call to the function.
  
\begin{quote}

\ntermdef{Expr1}

\defspace \opt{\nterm{SimpleExpr} .} \nterm{Identifier} = \nterm{Expr}

\defspace \nterm{SimpleExpr} \texttt{<- }\nterm{State}

\defspace \nterm{SimpleExpr} \texttt{<{}<-} \nterm{State}

\defspace \keyw{match} ( \nterm{InfixExpr} ) \{ \seq{CaseClause} \}

\defspace \keyw{atomic} \nterm{MetaArgs}  BlockExpr

\defspace \keyw{split} \nterm{MetaArgs}  BlockExpr

\defspace \keyw{unpack} BlockExpr

\defspace \nterm{InfixExpr}

\end{quote}

The assignment form is for fields or for already-declared local
variables, which must have been declared using \keyw{var}. 

The state change operator \texttt{<-} modifies the object to the left of the
arrow as follows:

\begin{itemize}

\item
All tags on the right are added to the object.  Old tags are kept
unless they are inconsistent with the new tags, i.e. the old tag and
new tag are (transitively) different cases of the same state.

\item
All members that were declared from tags being removed, are removed from
the object.

\item
All members on the right are added to the object.  All old members on
the left that are not explicitly removed according to the bullet
above, are retained.  

\item 
Futher details on the semantics of state change can be found in \citet{sunshine2011}.

\end{itemize}

The replacement operator \texttt{<{}<-} removes all tags and members from the object on the left and adds all tags and members of the state on the right.   

The type of either state change operations is \keyw{void}.

The \keyw{match} expression matches an input expression to one of
several cases using the \nterm{CaseClause} construct defined below.
The overall match expression evaluates to whatever value the
chosen case body evaluates to.

The \keyw{atomic} expression provides a save access environment to all
shared objects which belong to the data groups mentioned by the
\keyw{atomic} block. For a full definition of the semantics refer to
\citet{stork09:concurrency_by_default, stork10:uaeminium_spec}.

The \keyw{split} executes all statements of its body concurrently. To
allow parallel access to shared data the \keyw{split} block will split
the declared data group permissions into shared permissions, one for
each statement. For a full definition of the semantics refer to
\cite{stork09:concurrency_by_default, stork10:uaeminium_spec}.

The \keyw{unpack} is used to trade the group/access permission to the
specified object to gain access to the inner/nested groups declared
inside the object. For a full definition of the semantics refer to
\cite{stork09:concurrency_by_default, stork10:uaeminium_spec}.

\begin{quote}

\ntermdef{CaseClause}

\defspace \keyw{case} \nterm{Pattern} \nterm{BlockExpr}

\defspace \keyw{default} \nterm{BlockExpr}

\ntermdef{Pattern}

\defspace \nterm{QualifiedIdentifier}


%\defspace \keyw{default}


\ntermdef{QualifiedIdentifier}

\defspace \nterm{Identifier} \seq{ . \nterm{Identifier}}

\end{quote}

The value is matched against each of the cases in order.  For the
first case that matches, the corresponding expression list is
evaluated.  If no pattern matches, an exception is
thrown.

The first kind of pattern syntax tests the value's tags against
the \nterm{QualifiedIdentifier} given.  The match succeeds if
one of the tags of the value is equal to the tag
\nterm{QualifiedIdentifier}, or if one of the tags of the value
was declared in a state that is a transitive case of the
\nterm{QualifiedIdentifier} specified.

The \nterm{QualifiedIdentifier}
must resolve to a state declared with the \keyw{state}
keyword; otherwise, an exception is
thrown. 

For the default pattern, the match always succeeds.  If
there is a default pattern, it must be the last one in the match
expression.

\begin{quote}

\ntermdef{InfixExpr}

\defspace \nterm{SimpleExpr}

\defspace \nterm{CastExpr}

\defspace \nterm{InfixExpr} \nterm{IdentifierOrOperator} \nterm{InfixExpr}


\ntermdef{IdentifierOrOperator}

\defspace \nterm{Identifier} \alt \nterm{Operator}


\ntermdef{CastExpr}

\defspace \nterm{SimpleExpr} \opt{\keyw{as} \nterm{Type}}

\end{quote}

The operators defined in Java have the same precedence in Plaid as
they do in Java, except the ternary operator and right shift operators 
which are unsupported.  Identifiers as well as symbolic operators can be
used as infix operators; both are treated as method calls on the
object on the left of the operator.  Non-Java operators and
identifiers used as infix operators have a precedence above assignment
and state change, and below all other operators.

Cast expressions assert that a variable has a given type, and
also assert the relevant permission for that variable.  These casts
are trusted by the typechecker, but unchecked. A program that executes 
an invalid cast may fail at may fail at any point later in the program's 
execution.


\begin{quote}

\ntermdef{SimpleExpr}

\defspace \nterm{BlockExpr}

\defspace \keyw{new} \nterm{State}

\defspace \nterm{SimpleExpr2}

\end{quote}

The \keyw{new} statement creates an object initialized according to the
\nterm{State} specification given (defined below).

\begin{quote}

\ntermdef{BlockExpr}

\defspace \{ \opt{\nterm{StmtListSemi}} \}

\ntermdef{StmtListSemi}

\defspace \nterm{Stmt} \seq{ ; \nterm{Stmt}} \opt{;}

\end{quote}

Block expressions have a semicolon-separated list of statements, with
an optional semicolon at the end.  The statement list evaluates to the
value given by the last statement in the list.

\begin{quote}

\ntermdef{SimpleExpr2}

\defspace \nterm{SimpleExpr1}

\defspace \nterm{SimpleExpr2} \nterm{BlockExpr}

\end{quote}

To enable control structures with a natural, Java-like syntax, we allow
a function to be invoked passing a block expression as an argument.  The
block expression in this case is a zero-argument lambda.

\begin{quote}

\ntermdef{SimpleExpr1}

\defspace \nterm{Literal}

\defspace \nterm{Identifier}

\defspace \keyw{this}

\defspace ( \nterm{ExprList} ) 

\defspace \nterm{SimpleExpr1} . \nterm{Identifier}

\defspace \nterm{SimpleExpr1} . \keyw{new}

\defspace \nterm{SimpleExpr1}  \opt{\nterm{MetaArgs}} \nterm{ArgumentExpr}

\ntermdef{ExprList}

\defspace \nterm{Expr} \seq{ , \nterm{Expr}}


\ntermdef{ArgumentExpr}


\defspace (  \opt{\nterm{ExprList}} ) 

\end{quote}

\keyw{this} represents the receiver of a method call as in Java and is 
bound in method bodies declared as members of states.  Unlike Java,
\keyw{this} is not bound in field initializers.

Expressions can appear within parenthesis as a comma
separated list representing a tuple.

Java constructors can be invoked by calling \keyw{new}
on the Java class name.

Function and method invocation are handled uniformly by
supplying the arguments as a tuple.  Applications can be
chained, supporting currying.  Polymorphic arguments
are specified at each call site as well.

\section{Polymorphism}

\begin{quote}

\ntermdef{MetaArgsSpec}

\defspace < \nterm{MetaArgSpec} \opt{, \nterm{MetaArgSpec}} >

\ntermdef{MetaArgSpec}

\defspace \keyw{group} \opt{\nterm{GroupPermission}} \nterm{Identifier}

\ntermdef{GroupPermission}

\defspace \keyw{exclusive}

\defspace \keyw{shared}

\defspace \keyw{protected}

\ntermdef{MetaArgs}

\defspace < \nterm{SimpleExpr1} \opt{, \nterm{SimpleExpr1}} >

\end{quote}

Plaid supports polymorphism for data groups\footnote{Extending
  polymorphism to types should be straight forward.}. Plaid uses angle
bracket to enclose polymorphic parameters and arguments (similar to
Java's generics). A \emph{MetaArgSpec} describes a single
formal,polymorphic parameter. At the moment Paid only supports only
group parameters. A group parameter consists of the \keyw{group}
keyword to identify this parameter as group parameter, and optional
\emph{GroupPermission} (only optional for state declarations) and the
name of the parameter. For more information about data groups and
group parameters refer to \cite{stork09:concurrency_by_default,
  stork10:uaeminium_spec}.

\section{Declarations}

\begin{quote}
\ntermdef{DeclOrStateOp}

\defspace \nterm{Decl}

\defspace \nterm{StateOp}

\ntermdef{Decl}

\defspace \seq{\nterm{ModifierOrDefaultPermission}} \keyw{state} \nterm{Identifier}  \opt{\nterm{MetaArgs}}
          \opt{\keyw{case} \keyw{of} \nterm{QualifiedIdentifier}  \opt{\nterm{MetaArgs}}}
          \opt{\nterm{StateBinding}} \opt{;}

\defspace  \seq{\nterm{ModifierOrDefaultPermission}} \keyw{stateval} \nterm{Identifier}  \opt{\nterm{MetaArgs}}
          \opt{\nterm{StateBinding}} \opt{;}

\defspace \seq{\nterm{Modifier}} \nterm{MSpec} ;

\defspace \seq{\nterm{Modifier}} \nterm{MSpec} \nterm{BlockExpr}

\defspace \seq{\nterm{Modifier}} \nterm{FieldDecl} ;

\defspace \seq{\nterm{Modifier}} \nterm{GroupDecl} ;


\ntermdef{StateOp}

\defspace \keyw{remove} \nterm{Identifier} ;

\defspace \keyw{rename} \nterm{Identifier} \keyw{as} \nterm{Identifier} ;

\end{quote}

\keyw{state} and \keyw{stateval} declarations specify the implementation of a state,
as specified in the state definition. The \keyw{state} keyword means that this state is given its
own \textit{tag} that can be used to test whether objects are in that state.  Only states declared with \keyw{state} can be given in a pattern for a case in a \keyw{match} statement.

The \keyw{case} \keyw{of} keyword assigns a superstate. States have 
all of the members of a superstate. Different cases of the same superstate 
are orthogonal; no object may ever be tagged with two cases of the same superstate.

The final two declarations are for method and field declarations.  The
method declaration has a method header and an
optional method body.  If the body is missing then
the method is abstract and must be filled in by sub-states or when the
state is instantiated.

Fields and state operators are discussed in more detail below.

\begin{quote}

\ntermdef{StateBinding}

\defspace = \nterm{State}

\defspace \{ \seq{\nterm{Decl}} \}

\end{quote}

\begin{quote}

\ntermdef{State}

\defspace \nterm{StatePrim} \seq{\keyw{with} \nterm{StatePrim}} % left-associative

\ntermdef{StatePrim}

\defspace \nterm{SimpleExpr1} \opt{\{ \seq{\nterm{DecOrStateOp}} \}}

\defspace \{ \seq{\nterm{Decl}} \}

\defspace \keyw{freeze} \nterm{SimpleExpr1}

\end{quote}

A state is a composition of primitive states separated by the
\keyw{with} keyword.  These primitive states include literal blocks
with a series of declarations and references to some
previous object or state definition. In addition, objects can be transformed into primitive states
with the \keyw{freeze} keyword. 

Composition is in general symmetric, as in traits.  It is an error if two states are composed with a member in common. The conflict can be resolved manually with state operators, remove members from, and rename members in a state. 



\begin{quote}


%%%%%%%%%%%%%%%%%% Fields and Methods %%%%%%%%%%%%%%%%%%%%%

\ntermdef{GroupDecl}

\defspace \keyw{group} \nterm{Identifier} = \keyw{new} \keyw{group} ;

\ntermdef{FieldDecl}

\defspace \nterm{ConcreteFieldDecl}

\defspace \nterm{AbstractFieldDecl}

\ntermdef{ConcreteFieldDecl}

\defspace \opt{\nterm{Specifier}} \opt{Type} \nterm{Identifier} = \nterm{Expr}

\ntermdef{AbstractFieldDecl}

\defspace \nterm{Specifier} \nterm{Identifier} 

\defspace \opt{Specifier} \nterm{Type} \nterm{Identifier}




\ntermdef{MSpec}

\defspace \keyw{method} \opt{\nterm{Type}} \nterm{IdentifierOrOperator} \opt{\nterm{MetaParams}} ( \opt{\nterm{Args}} )
          \opt{[ \nterm{Args} ]}

\ntermdef{Args}

\defspace \opt{\nterm{ArgSpec}} \nterm{Identifier} \seq{ , \opt{\nterm{ArgSpec}} \nterm{Identifier}}

\ntermdef{ArgSpec}

\defspace \nterm{Type} \opt{$>>$ \nterm{Type}}

\end{quote}

The \nterm{FieldDecl} form should be familiar from Java-like
languages.  If no expression is given then the field is abstract.  All
fields can only be assigned from within the
state.
When fields are first defined a specifier (\keyw{var} or \keyw{val})
must be given; later, when the field is overridden and given a concrete
value, the specifier may be omitted.
\keyw{var} fields are assignable, \keyw{val} fields
are not.

If a type is missing and an expression is given for a
  \keyw{val} field, then the type of the field is inferred from the
  expression as in variable declaration statements.  
  If the type is missing and either no expression is
  given or it is a \keyw{var} field, then the type is \keyw{dynamic}.

The method header \nterm{MSpec} also has a standard
format.  As in function declarations, programmers may
optionally include types for any variables in scope within  [].  For 
methods declared within states, the distinguished variable
\keyw{this}, representing the receiver of the method, may appear
in this list.  If it does not, then the type of the receiver defaults
to the structure representing the state the method is defined in
with the default permission for that state.  The receiver ends 
the method with the same type. 

Each argument specification includes the required type at the time of the
method call or function application.  If the parameter ends the call with a different
type this is indicated with a $>>$ and the resulting type. If no
resulting type is specified then it defaults to the required type.


\begin{quote}

%%%%%%%%%%%%%%%%%% Modifiers %%%%%%%%%%%%%%%%%%%%%

\ntermdef{Modifier}

\defspace \keyw{requires}

\defspace \keyw{override}

\ntermdef{DefaultPermission}

\defspace \keyw{immutable}

\ntermdef{ModifierOrDefaultPermission}

\defspace \nterm{Modifier}

\defspace \nterm{DefaultPermission}

\end{quote}


\keyw{override} indicates that a method overrides a function of the
same name during composition.

\keyw{requires} is similar to \keyw{abstract} in Java.  However,
things are more interesting in Plaid, because one can pass around an
object that has abstract/required members.  It is not necessary to
use the \keyw{requires} modifier in state definitions; one can simply
leave off the definition of a function.  \keyw{requires} is necessary
in types, however, to distinguish the presence vs. absence of a
member in that type.  Unlike in Java, methods may be called on an
object that has a required member, but only if the type given to the
method's receiver does not expect that member to be present.

\keyw{immutable} means a state is immutable and the permission of any fields,
local variables, or parameters declared to have the structure represented
by the state defaults to \keyw{immutable} when a permission
is not specified.  If the state is not immutable
then the default permission is \keyw{unique}.




\section{Types}

\begin{quote}

\ntermdef{Type}

\defspace \keyw{void}

\defspace \opt{\keyw{immutable}} \nterm{LambdaStructure}

\defspace \opt{\nterm{Permission}} \nterm{NominalStructure}

\defspace (\nterm{Type})

\ntermdef{Permission}

\defspace \keyw{unique}

\defspace \nterm{SymmetricPermission}

\defspace \nterm{LocalPermission}

\ntermdef{SymmetricPermission}

\defspace \keyw{shared} <\nterm{SimpleExpr1}>

\defspace \keyw{immutable}

\ntermdef{LocalPermission} 

\defspace \keyw{local} \nterm{SymmetricPermission}

\ntermdef{LambdaStructure} 

\defspace \opt{\nterm{MetaParams}} (\nterm{ArgSpecs}) \opt{[ \nterm{Args}]} -> \nterm{Type}

\ntermdef{NominalStructure} 

\defspace \keyw{top}

\defspace \nterm{QualifiedIdentifier} \opt{MetaArgs}

\ntermdef{ArgSpecs}

{\defspace \nterm{ArgSpec} \seq{ * \nterm{ArgSpec} }}
\end{quote}

All types in Plaid include a permission and a structure.  The most general
type is \keyw{void} which represents the weakest
permission, \keyw{none} (not expressible in the source), with the most 
general structure, \keyw{top}. References may be inferred to have the
type \keyw{dynamic}.  Uses of values with type \keyw{dynamic} are not statically
guaranteed to be type-safe.  An unsafe cast must appear in the source
for a \keyw{dynamic} value to be used in statically typed code. 
All other types are written as an optional \nterm{Permission} and a structure.

A \nterm{LambdaStructure} represents a function structure.  They optionally
include the initial and resulting types of references in scope during a
function call list in [ ] as in function and method declarations. 
Formally, a function that accepts multiple arguments actually accepts an
argument tuple, which is written with a \code{*}-separated list.

A \nterm{NominalStructure} represents the structure 
given by a declared state
or the distinguished \keyw{top} structure, which is a superstructure
of all structures.  If the state represented by the \nterm{NominalStructure}
is polymorphic, then \nterm{MetaArgs} must be provided.

If the permission for a structure is not given, then a default is inferred.
A \nterm{LambdaStructure} can only be declared with 
an \keyw{immutable} permission and defaults to \keyw{immutable}
if a permission is not given.  The \keyw{top} structure 
defaults to the \keyw{none} permission and a \nterm{NominalStructure} defaults 
to the \keyw{immutable} permission if it represents an 
\keyw{immutable} state and to \keyw{unique} otherwise. 

The \keyw{unique} permission indicates that there are no
usable aliases to the same object.  There may be other
references to the object with the \keyw{none} permission
which does not allow the object to be used in any way.

A \nterm{SymmetricPermission} allow new aliases to be 
created with the same permission.
\keyw{immutable} references cannot be used to update the object
but can assume that no other references make changes. 
\keyw{shared} references can make changes, but must assume that other
references may have changed the object.

A \keyw{local} permission gives the same abilities
and guarantees as its underlying \nterm{SymmetricPermission}, 
but is restricted to local variables and parameters.  
A \keyw{local} reference cannot be assigned into a field.
This restriction allows \keyw{local} permissions to be
returned to their original location when their reference goes out of 
scope.  This may allow the original reference to regain a stronger permission.
For example, a \keyw{unique} reference used as a function parameter
that requires and results in a \keyw{local} permission will
still be \keyw{unique} after the function call.



\section{Compilation Units}

\begin{quote}

\ntermdef{CompilationUnit}

\defspace \keyw{package} \nterm{QualifiedIdentifier} \code{;} \nterm{Decls}

\end{quote}

A compilation unit is made up of a (required) package clause followed
by a sequence of declarations.

\pII{
\subsection{New Import Design (overrides below)}
%
Goals
 * modules should be parametric in their imports by default
%
 * programming in an extensible world should look like, and be no more
 costly than, programming in the Java style (packages and imports)
%
 * if you have security constraints and don't want things to be overridden,
there should be a (not too painful) way to enforce that
%
the import statement looks like Java, but means requires a module with
the right signature.
%
by default, an import gets resolved to the actual implementation module
named.
%
there is a mechanism for redirecting the import to some other module
that conforms to the signature (this is done in a surrounding module
or some kind of package mechanism)
%
there is a mechanism for sealing imports (really just binding them to
their defaults) to a module or to a package so they cannot be rebound
externally.
}
\subsection{Imports}

\begin{quote}

\ntermdef{Decl}

\defspace \keyw{import} \nterm{QualifiedIdentifier} \opt{\nterm{DotStar}} \code{;}

\ntermdef{DotStar}

\defspace \code{. *}

\end{quote}

An import statement imports a qualified name into the current scope so
it can be referred to by the last identifier in the qualified name.
If the import ends in .*, then all the members of the given \nterm{Name} are
imported into the current scope.

As in Java, importing the same simple name twice is an error unless
the fully qualified name is the same.  Importing a specific simple
name always overrides importing all elements of a package where
that name is defined, regardless of which definition goes first.
In general, Plaid follows the Java Language Specification section
7.5.

\TODO{Need to define imports more precisely!  look at Java.  Address
conflicts between local names and imports.}

\subsection{Java Interoperability}

\minisec{Accessing Java from Plaid.}  Any java package, class, or
class member can be referred to via a qualified name.  Imported
name(s) can include a package, class, or class member from Java.
Instances of a Java class C may be created by invoking C.new(...)  and
passing appropriate arguments for one of the constructors of class C.
A static method m of C may be invoked with the syntax C.m(...).  An
instance methods of a Java object o may be invoked with the syntax
o.m(...).  Arguments passed to calls of Java constructors and methods
may be Java objects.  Plaid integers, strings, and booleans are
converted to appropriate Java primitive, String, and numeric object
types (e.g. java.lang.Integer) depending on the declared type of the
method's formal parameters.  If a Java method takes an Object or
plaid.runtime.PlaidObject as an argument, then a Plaid object can be
passed to it, allowing Java code to access Plaid objects.

\minisec{Implementing Java Interfaces.}  A Plaid state can be declared to
be a case of a Java interface.  In that case, any \keyw{new}
expression that creates an object with that state will generate a
Plaid object that extends the appropriate Java interface.  The Plaid
object may then be passed to a Java method that takes the interface
type as an argument.  Methods of the interface that are invoked by
Java are converted into calls to Plaid methods of the same name and
arguments, as described immediately below.

\minisec{Accessing Plaid from Java.}  Java code may invoke methods of
Plaid objects when those objects implement Java interfaces, as
decribed above, or reflectively through the plaid.runtime.PlaidObject
interface.  When calling a Plaid method through this interface, Java
objects of type Integer, String, Booleans, and other numeric objects
are converted into the corresponding Plaid types.  PlaidObjects and
Java objects are passed through unchanged, and their methods may be
invoked from Plaid in the usual way described above.  The detailed
interface of plaid.runtime.PlaidObject is specified in the javadoc
for that interface.

\TODO{Clean interoperability between the null value in Java and
the Plaid type system}

\subsection{File System Conventions}

Plaid uses the Java classpath mechanism to find files.  When searching
for a definition for a qualified name $x_1$.$x_2 \ldots x_n$, where $x_1$ is not in
scope, the system will search for a directory under the classpath
named $x_1$ and then look for a file named $x_2$ there.  If the file is a
directory, the search proceeds with $x_3$ and so forth.

A compilation unit is stored in a file with extension .plaid. The file must be stored in a directory in the class path that corresponds to the package. For example, all $x_1$.$x_2$ package must be stored in \$CLASSPATH\$/$x_1$/$x_2$/.

For each top-level declaration in the file, a Java class in the package declared
is created with the name of the top-level declaration.  The Java class
implementing a declaration is found at run time using Java's normal
classpath-based lookup mechanism.

Each compilation unit may have one top-level declaration that has the same
name as the file name (without the .plaid extension).  This declaration
is the only one that is visible from other files. All other declarations in the file are private.

Other public declarations may be placed in a special file named package.plaid. There may be one package.plaid file per package. All declarations in this file are public.

\TODO{later: all other decls are private}




\subsection{Applications}

An application is any globally-visible function that takes no arguments.

\noindent
If the user types at the command line:

\begin{quote}
\cmdline{plaid Name}
\end{quote}

\noindent
where \cmdline{Name} is a qualified name, the \cmdline{plaid}
executable will search the classpath for a declaration of the named
function and will try to execute it.


\pII{
\subsection{IDE considerations}
%
IDEs should show state changes where they occur in code.  For example:
%
\begin{quote}
\code{file.close()       file>>ClosedFile}
\end{quote}
}


\bibliographystyle{plainnat}
\bibliography{spec}


\end{document}
